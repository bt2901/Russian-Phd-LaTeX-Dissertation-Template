\begin{frame}[t]{Рецепты моделирования}
Инициализация модели в Baseline Recipe\\

\footnotesize \texttt{\\
\textcolor{blue}{topics}: \\
\ \ \textcolor{blue}{specific\_topics}: \colorbox{gray!30}{\{specific\_topics\}} \\
\ \     \textcolor{blue}{background\_topics}: \colorbox{gray!30}{\{background\_topics\}} \\
\textcolor{blue}{regularizers}: \\
\ \ - DecorrelatorPhiRegularizer: \\
\ \ \ \ \textcolor{blue}{name}: decorrelation\_phi \\
\ \ \ \ \textcolor{blue}{topic\_names}: specific\_topics \\
\ \ \ \ \textcolor{blue}{class\_ids}: \colorbox{gray!30}{\{modality\_list\}} \\
\ \ - SmoothSparsePhiRegularizer: \\
\ \ \ \ \textcolor{blue}{name}: smooth\_phi\_bcg \\
\ \ \ \ \textcolor{blue}{topic\_names}: background\_topics \\
\ \ \ \ \textcolor{blue}{class\_ids}: \colorbox{gray!30}{\{modality\_list\}} \\
\ \ \ \ \textcolor{blue}{tau}: 0.1 \\
\ \ \ \ \textcolor{blue}{relative}: true \\
\ \ - SmoothSparseThetaRegularizer: \\
\ \ \ \ \textcolor{blue}{name}: smooth\_theta\_bcg \\
\ \ \ \ \textcolor{blue}{topic\_names}: background\_topics \\
\ \ \ \ \textcolor{blue}{tau}: 0.1 \\
\ \ \ \ \textcolor{blue}{relative}: true \\
}

\end{frame}

\begin{frame}[t]{Рецепты моделирования}
Перебор коэффициента декорреляции в Baseline Recipe\\

\footnotesize \texttt{\\
\textcolor{blue}{stages}: \\
- RegularizersModifierCube: \\
\ \ \textcolor{blue}{num\_iter}: 20 \\
\ \ \textcolor{blue}{reg\_search}: add \\
\ \ \textcolor{blue}{regularizer\_parameters}: \\
\ \ \ \ \textcolor{blue}{name}: decorrelation\_phi \\
\ \ \textcolor{blue}{selection}: \\
\ \ \ \ - PerplexityScore@all < 1.05 * MINIMUM(PerplexityScore@all) and BleiLaffertyScore -> max \\
\ \ \textcolor{blue}{strategy}: PerplexityStrategy \\
\ \ \textcolor{blue}{strategy\_params}: \\
\ \ \ \ \textcolor{blue}{start\_point}: 0 \\
\ \ \ \ \textcolor{blue}{step}: 0.01 \\
\ \ \ \ \textcolor{blue}{max\_len}: 50 \\
\ \ \textcolor{blue}{tracked\_score\_function}: PerplexityScore@all \\
\ \ \textcolor{blue}{use\_relative\_coefficients}: true	\\
}

Пространство поиска: все возможные $\tau$ у регуляризатора с названием \texttt{decorrelation\_phi}.\\

Стратегия обхода: начинаем с $\tau=0$, увеличиваем коэффициент линейно на 0.01 до тех пор, пока перплексия новой модели не будет слишком большой.
\end{frame}




\begin{frame}{Репараметризация сглаживания/разреживания}
	
\begin{equation}
\tau = \frac{n}{|D| \cdot |T|} \frac{\lambda}{(1-\lambda)}
\end{equation}

\begin{equation}
\tau = \frac{n}{|W|\cdot|T|} \frac{\lambda}{(1-\lambda)}  
\end{equation}

Интерпретация: берём $\lambda$ частей априорного распределения
\[
\frac{1}{|T|}\text{ или }\frac{1}{|W|}
\]
и $(1-\lambda)$ частей от оценки максимума правдоподобия 
\[
\frac{n_{td}}{n_d}\text{ или }\frac{n_{wt}}{n_t}
\]

\end{frame}
