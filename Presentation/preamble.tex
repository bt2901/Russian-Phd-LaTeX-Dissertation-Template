\begin{frame}[noframenumbering,plain]
    \setcounter{framenumber}{1}
    \maketitle
\end{frame}

\begin{frame}
    \frametitle{Положения, выносимые на защиту}
\small
    \begin{itemize}
\item
    Методология построения тематических моделей, обеспечивающая формирование <<рецептов моделирования>> с автоматизированным подбором гиперпараметров по множеству критериев и отличающаяся использованием относительных коэффициентов регуляризации и кубов гиперпараметров.
\item
    Архитектура библиотеки TopicNet, обеспечивающая программную реализацию данной методологии и отличающаяся использованием удобного языка описания кубов гиперпараметров и возможностью создания пользовательских регуляризаторов и метрик качества на языке Python.
\item
    Универсальный рецепт моделирования, обеспечивающий многокритериальный выбор тематических моделей для широкого класса задач, отличающийся предварительной настройкой куба гиперпараметров по набору разнородных задач тематического моделирования.
\item
    Программная реализация нового критерия когерентности, обеспечивающая его эффективное вычисление и отличающаяся более полным использованием данных о сочетаемости слов внутри текстовых документов.
%\item
%    Программная реализация псевдорегуляризатора в библиотеке TopicNet, обеспечивающего быстрое однопроходное вычисление тематических векторных представлений документов и улучшение качества тематической модели по множеству критериев.
    \end{itemize}

    
\end{frame}
\note{
    Проговариваются вслух положения, выносимые на защиту
}

\begin{frame}
    \frametitle{Содержание}
    \tableofcontents
\end{frame}
\note{
    Работа состоит из четырёх глав.

    \medskip
    В первой главе \dots

    Во второй главе \dots

    Третья глава посвящена \dots

    В четвёртой главе \dots
}
