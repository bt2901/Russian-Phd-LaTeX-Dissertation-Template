% Общие стили оформления.
% Возможные варианты значений ищите в описании библиотеки beamer
\usetheme{Pittsburgh}
\usecolortheme{whale}

% \usetheme[secheader]{Boadilla}
% \usecolortheme{seahorse}

% Размер полей слайдов
\setbeamersize{text margin left=1cm,%
               text margin right=1cm}

% выключение кнопок навигации
\beamertemplatenavigationsymbolsempty

% Размеры шрифтов
\setbeamerfont{title}{size=\large}
\setbeamerfont{subtitle}{size=\small}
\setbeamerfont{author}{size=\normalsize}
\setbeamerfont{institute}{size=\small}
\setbeamerfont{date}{size=\normalsize}
\setbeamerfont{bibliography item}{size=\small}
\setbeamerfont{bibliography entry author}{size=\small}
\setbeamerfont{bibliography entry title}{size=\small}
\setbeamerfont{bibliography entry location}{size=\small}
\setbeamerfont{bibliography entry note}{size=\small}
% Аналогично можно настроить и другие размеры.
% Названия классов элементов можно найти здесь
% http://www.cpt.univ-mrs.fr/~masson/latex/Beamer-appearance-cheat-sheet.pdf

% Цвет элементов
\setbeamercolor{footline}{fg=blue}
\setbeamercolor{bibliography item}{fg=black}
\setbeamercolor{bibliography entry author}{fg=black}
\setbeamercolor{bibliography entry title}{fg=black}
\setbeamercolor{bibliography entry location}{fg=black}
\setbeamercolor{bibliography entry note}{fg=black}
% Аналогично можно настроить и другие цвета.
% Названия классов элементов можно найти здесь
% http://www.cpt.univ-mrs.fr/~masson/latex/Beamer-appearance-cheat-sheet.pdf

% Нумеровать список статей
% https://tex.stackexchange.com/a/419506/104425
\setbeamertemplate{bibliography item}{\insertbiblabel}
% или убрать номера
% \setbeamertemplate{bibliography item}{}

% Использовать шрифт с засечками для формул
% https://tex.stackexchange.com/a/34267/104425
\usefonttheme[onlymath]{serif}

% https://tex.stackexchange.com/a/291545/104425
\makeatletter
\def\beamer@framenotesbegin{% at beginning of slide
    \usebeamercolor[fg]{normal text}
    \gdef\beamer@noteitems{}%
    \gdef\beamer@notes{}%
}
\makeatother

% footer презентации
\setbeamertemplate{footline}{
    \leavevmode%
    \hbox{%
        \begin{beamercolorbox}[wd=.333333\paperwidth,ht=2.25ex,dp=1ex,center]{}%
            % И. О. Фамилия, Организация кратко
            \thesisAuthorShort, \thesisOrganizationShort
        \end{beamercolorbox}%
        \begin{beamercolorbox}[wd=.333333\paperwidth,ht=2.25ex,dp=1ex,center]{}%
            % Город, 20XX
            \thesisCity, \thesisYear
        \end{beamercolorbox}%
        \begin{beamercolorbox}[wd=.333333\paperwidth,ht=2.25ex,dp=1ex,right]{}%
            Стр. \insertframenumber{} из \inserttotalframenumber \hspace*{2ex}
        \end{beamercolorbox}}%
    \vskip0pt%
}

% вывод на экран заметок к презентации
\ifnumequal{\value{presnotes}}{0}{}{
    \setbeameroption{show notes}
    \ifnumequal{\value{presnotes}}{2}{
        \setbeameroption{show notes on second screen=\presposition}
    }{}
}

\newcommand{\norm}{\mathop{\mathrm{norm}}\limits}
\usepackage{multirow}
\usepackage{cmap}
\usepackage{multicol}

\newcommand{\cond}{\mspace{2mu}{|}\mspace{2mu}}

\usepackage{listings}
\usepackage{tabularx}

\lstset{
  %basicstyle=\itshape,
  %basicstyle=\ttfamily\color{black},
  basicstyle=\rmfamily\color{black},
  xleftmargin=3em,
  literate={->}{$\rightarrow$}{2}
           {<}{$\langle$}{1}
           {>}{$\rangle$}{1}
           {to}{{\textbf{{\color{red}\textemdash$>$}}}}{2}
           {less}{{{\color{red}$<$}}}{1}
           {great}{{{\color{red}$>$}}}{1}
           {eq}{{{\color{red}$=$}}}{1}
           {model}{{{\color{red}\texttt{model.}}}}{6}
           {MAX}{{{\color{red}\texttt{MAX}}}}{3}
           {MIN}{{{\color{red}\texttt{MIN}}}}{3}
           {AND}{{{\color{red}\texttt{AND}}}}{3}
           {COLLECT}{{{\color{red}\texttt{COLLECT}}}}{7}
}


\newcommand{\BeginSectionSplash}[1]{
    \section{#1}
    \begin{frame}[plain, noframenumbering]
        \begin{center}
            \Huge
            #1
        \end{center}
    \end{frame}
}
\newcommand{\BeginSection}[1]{
    \section{#1}
}



\usetikzlibrary{quotes,arrows.meta}
\tikzset{
  annotated cuboid/.pic={
    \tikzset{%
      every edge quotes/.append style={midway, auto},
      /cuboid/.cd,
      #1
    }
    \draw [every edge/.append style={pic actions, densely dashed, opacity=.5}, pic actions]
    (0,0,0) coordinate (o) -- ++(-\cubescale*\cubex,0,0) coordinate (a) -- ++(0,-\cubescale*\cubey,0) coordinate (b) edge coordinate [pos=1] (g) ++(0,0,-\cubescale*\cubez)  -- ++(\cubescale*\cubex,0,0) coordinate (c) -- cycle
    (o) -- ++(0,0,-\cubescale*\cubez) coordinate (d) -- ++(0,-\cubescale*\cubey,0) coordinate (e) edge (g) -- (c) -- cycle
    (o) -- (a) -- ++(0,0,-\cubescale*\cubez) coordinate (f) edge (g) -- (d) -- cycle;
    \path [every edge/.append style={pic actions, |-|}]
    (b) +(0,-5pt) coordinate (b1) edge ["\annotx"'] (b1 -| c)
    (b) +(-5pt,0) coordinate (b2) edge ["\annoty"] (b2 |- a)
    (c) +(3.5pt,-3.5pt) coordinate (c2) edge ["\annotz"'] ([xshift=3.5pt,yshift=-3.5pt]e)
    ;
  },
  /cuboid/.search also={/tikz},
  /cuboid/.cd,
  xstring/.store in=\annotx,
  ystring/.store in=\annoty,
  zstring/.store in=\annotz,
  width/.store in=\cubex,
  height/.store in=\cubey,
  depth/.store in=\cubez,
  units/.store in=\cubeunits,
  scale/.store in=\cubescale,
  width=10,
  height=10,
  depth=10,
  units=cm,
  scale=.1,
}
