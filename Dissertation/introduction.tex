\chapter*{Введение}                         % Заголовок
\addcontentsline{toc}{chapter}{Введение}    % Добавляем его в оглавление

\newcommand{\actuality}{}
\newcommand{\progress}{}
\newcommand{\aim}{{\textbf\aimTXT}}
\newcommand{\tasks}{\textbf{\tasksTXT}}
\newcommand{\novelty}{\textbf{\noveltyTXT}}
\newcommand{\thInfluence}{\textbf{\thInfluenceTXT}}
\newcommand{\prInfluence}{\textbf{\prInfluenceTXT}}
\newcommand{\methods}{\textbf{\methodsTXT}}
\newcommand{\defpositions}{\textbf{\defpositionsTXT}}
\newcommand{\reliability}{\textbf{\reliabilityTXT}}
\newcommand{\probation}{\textbf{\probationTXT}}
\newcommand{\contribution}{\textbf{\contributionTXT}}
\newcommand{\publications}{\textbf{\publicationsTXT}}


{\actuality} Тематическое моделирование --- метод анализа текстов, производящий мягкую кластеризацию как слов, так и документов. У тематического моделирования есть два важных свойства, отличающих его от других методов работы с текстами. 

Во-первых, тренировка модели происходит ``без учителя'': для того, чтобы обработать корпус текстов посредством тематического моделирования не требуется размеченных данных или предобученных моделей. Это позволяет использовать тематическое моделирование в задачах, использование в которых предобученных моделей затрудненно: для анализа узкоспецифических текстов или текстов на редких языках, для анализа тексто-подобных данных (программный код, тексты песен, банковские транзакции, географические данные, музыкальные произведения).

Во-вторых, результатом работы тематической модели являются \textit{темы}, описываемые вероятностными распределениями. Компоненты этих распределений --- вероятность слова с учётом темы и вероятность темы в документе --- прозрачны и играют понятную роль в модели. 
Зачастую эксперт может связать тему (как множество слов и множество документов) с каким-либо понятием из предметной области (т.е. проинтерпретировать её); такое осознание помогает пониманию структуры коллекции. Поэтому важное применение тематического моделирования --- помощь в понимании больших массивов неструктурированных данных.

Аддитивная Регуляризация Тематических это такой подход который даёт много гибкости.

\textbf{Степень разработанности темы исследования}. 

Подбор гиперпараметров. Доступность. Подстраховка от заблуждений.

Всё это является нерешёнными проблемами.

% {\progress}
% Этот раздел должен быть отдельным структурным элементом по
% ГОСТ, но он, как правило, включается в описание актуальности
% темы. Нужен он отдельным структурынм элемементом или нет ---
% смотрите другие диссертации вашего совета, скорее всего не нужен.

{\aim} данного диссертационного исследования является разработка методов построения интерпретируемых тематических моделей, применимых для широкого ряда задач. Проделанная работа опубликована на GitHub как открытая библиотека TopicNet.

Для~достижения поставленной цели решаются следующие {\tasks}:
\begin{enumerate}[beginpenalty=10000] % https://tex.stackexchange.com/a/476052/104425
  \item кластеризация интентов при помощи специального куба
  \item изучение принятых метрик интерпретируемости
  \item введение кастомных метрик качества
  \item введение кастомных регуляризаторов
  \item перформанс бэзлайна сравнивается с генсимом и рецептом Мурата
\end{enumerate}

{\novelty}
\begin{enumerate}[beginpenalty=10000] % https://tex.stackexchange.com/a/476052/104425
  \item Впервые \ldots
  \item Впервые \ldots
  \item Было выполнено оригинальное исследование \ldots
\end{enumerate}

{\influence} 
Абзац про нужность библиотеки ТопикНет

Абзац про пользу относительных коэффициентов

Абзац про применимость всего этого в таких-то задачах.


{\methods} В работе использованы методы теории вероятностей, оптимизации, NLP. Экспериментальное исследование проводится на языке Python; опубликованная на GitHub библиотека TopicNet, подытоживающая результаты исследования открыта для широкой публики и удовлетворяет принципам воспроизводимости результатов.

{\defpositions}
\begin{enumerate}[beginpenalty=10000] % https://tex.stackexchange.com/a/476052/104425
  \item Изучение репрезентативности имеющихся мер интерпретируемости
  \item Предложены относительные веса модальностей, и относительные коэффициенты сглаживания/разреживания. Обеспечивают переносимость тематических моделей. Отличаются от аналогов...
  \item Псевдорегуляризатор, обеспечивающий быстрое однопроходное вычисление векторизации документов. Отличается от аналогов тем, что улучшает многие меры качества.
  \item Концепция дерева эксперимента и кубов в библиотеке TopicNet
  \item Новая адаптивная стратегия регуляризации, реализованная как отдельный куб в TopicNet
\end{enumerate}


{\reliability} полученных результатов обеспечивается \ldots 

{\probation}
Основные результаты диссертации докладывались на следующих конференциях и семинарах:
\begin{itemize}
    \item Международная конференция по компьютерной лингвистике “Диалог”, Москва, 1 июня 2018.
    \item International Conference Recent Advances in Natural Language Processing (RANLP), Варна, 3 сентября 2019.
    \item Открытая лекция в рамках образовательного проекта Физтех.Рост, Долгопрудный, 18 октября 2019.
    \item Научный семинар про коэффициентов
    \item Научный семинар про ТопикНет
    \item OpenTalks.AI – ведущая независимая открытая конференция по искусственному интеллекту, Москва, 20 февраля 2020 года.
    \item International Conference on Language Resources and Evaluation (LREC), Марсель (должна была состоятся в мае 2020).
\end{itemize}

{\contribution} Автор принимал активное участие \ldots

\ifnumequal{\value{bibliosel}}{0}
{%%% Встроенная реализация с загрузкой файла через движок bibtex8. (При желании, внутри можно использовать обычные ссылки, наподобие `\cite{vakbib1,vakbib2}`).
    {\publications} Основные результаты по теме диссертации изложены
    в~XX~печатных изданиях,
    X из которых изданы в журналах, рекомендованных ВАК,
    X "--- в тезисах докладов.
}%
{%%% Реализация пакетом biblatex через движок biber
    \begin{refsection}[bl-author, bl-registered]
        % Это refsection=1.
        % Процитированные здесь работы:
        %  * подсчитываются, для автоматического составления фразы "Основные результаты ..."
        %  * попадают в авторскую библиографию, при usefootcite==0 и стиле `\insertbiblioauthor` или `\insertbiblioauthorgrouped`
        %  * нумеруются там в зависимости от порядка команд `\printbibliography` в этом разделе.
        %  * при использовании `\insertbiblioauthorgrouped`, порядок команд `\printbibliography` в нём должен быть тем же (см. biblio/biblatex.tex)
        %
        % Невидимый библиографический список для подсчёта количества публикаций:
        \ifxetexorluatex\selectlanguage{english}\fi
        \printbibliography[heading=nobibheading, section=1, env=countauthorvak,          keyword=biblioauthorvak]%
        \printbibliography[heading=nobibheading, section=1, env=countauthorwos,          keyword=biblioauthorwos]%
        \printbibliography[heading=nobibheading, section=1, env=countauthorscopus,       keyword=biblioauthorscopus]%
        \printbibliography[heading=nobibheading, section=1, env=countauthorconf,         keyword=biblioauthorconf]%
        \printbibliography[heading=nobibheading, section=1, env=countauthorother,        keyword=biblioauthorother]%
        \printbibliography[heading=nobibheading, section=1, env=countregistered,         keyword=biblioregistered]%
        \printbibliography[heading=nobibheading, section=1, env=countauthorpatent,       keyword=biblioauthorpatent]%
        \printbibliography[heading=nobibheading, section=1, env=countauthorprogram,      keyword=biblioauthorprogram]%
        \printbibliography[heading=nobibheading, section=1, env=countauthor,             keyword=biblioauthor]%
        \printbibliography[heading=nobibheading, section=1, env=countauthorvakscopuswos, filter=vakscopuswos]%
        \printbibliography[heading=nobibheading, section=1, env=countauthorscopuswos,    filter=scopuswos]%
        %
        \nocite{*}\ifxetexorluatex\selectlanguage{russian}\fi%
        %
        \nocite{intracoh, popov_hier, bulatov2020topicnet, thetaless, prog_cook, prog_view}
        
        {\publications} Основные результаты по теме диссертации изложены в~\arabic{citeauthor}~печатных изданиях,
        \arabic{citeauthorvak} из которых изданы в журналах, рекомендованных ВАК\sloppy%
        \ifnum \value{citeauthorscopuswos}>0%
            , \arabic{citeauthorscopuswos} "--- в~периодических научных журналах, индексируемых Web of~Science и Scopus\sloppy%
        \fi%
        \ifnum \value{citeauthorconf}>0%
            , \arabic{citeauthorconf} "--- в~тезисах докладов.
        \else%
            .
        \fi%
        \ifnum \value{citeregistered}=1%
            \ifnum \value{citeauthorpatent}=1%
                Зарегистрирован \arabic{citeauthorpatent} патент.
            \fi%
            \ifnum \value{citeauthorprogram}=1%
                Зарегистрирована \arabic{citeauthorprogram} программа для ЭВМ.
            \fi%
        \fi%
        \ifnum \value{citeregistered}>1%
            Зарегистрированы\ %
            \ifnum \value{citeauthorpatent}>0%
            \formbytotal{citeauthorpatent}{патент}{}{а}{}\sloppy%
            \ifnum \value{citeauthorprogram}=0 . \else \ и~\fi%
            \fi%
            \ifnum \value{citeauthorprogram}>0%
            \formbytotal{citeauthorprogram}{программ}{а}{ы}{} для ЭВМ.
            \fi%
        \fi%
        % К публикациям, в которых излагаются основные научные результаты диссертации на соискание учёной
        % степени, в рецензируемых изданиях приравниваются патенты на изобретения, патенты (свидетельства) на
        % полезную модель, патенты на промышленный образец, патенты на селекционные достижения, свидетельства
        % на программу для электронных вычислительных машин, базу данных, топологию интегральных микросхем,
        % зарегистрированные в установленном порядке.(в ред. Постановления Правительства РФ от 21.04.2016 N 335)
    \end{refsection}%
    \begin{refsection}[bl-author, bl-registered]
        % Это refsection=2.
        % Процитированные здесь работы:
        %  * попадают в авторскую библиографию, при usefootcite==0 и стиле `\insertbiblioauthorimportant`.
        %  * ни на что не влияют в противном случае
        \nocite{intracoh}
        \nocite{popov_hier}
        \nocite{bulatov2020topicnet}
        \nocite{thetaless}
        \nocite{prog_cook}
        \nocite{prog_view}
        
    \end{refsection}%
        %
        % Всё, что вне этих двух refsection, это refsection=0,
        %  * для диссертации - это нормальные ссылки, попадающие в обычную библиографию
        %  * для автореферата:
        %     * при usefootcite==0, ссылка корректно сработает только для источника из `external.bib`. Для своих работ --- напечатает "[0]" (и даже Warning не вылезет).
        %     * при usefootcite==1, ссылка сработает нормально. В авторской библиографии будут только процитированные в refsection=0 работы.
}


% Для добавления в список публикаций автора работ, которые не были процитированы в
% автореферате, требуется их~перечислить с использованием команды \verb!\nocite! в
% \verb!Synopsis/content.tex!.
 % Характеристика работы по структуре во введении и в автореферате не отличается (ГОСТ Р 7.0.11, пункты 5.3.1 и 9.2.1), потому её загружаем из одного и того же внешнего файла, предварительно задав форму выделения некоторым параметрам

\textbf{Объем и структура работы.} Диссертация состоит из~введения, двух обзорных глав, четырёх глав с результатами проведенного исследования, заключения, списка литературы и приложения.
%% на случай ошибок оставляю исходный кусок на месте, закомментированным
%Полный объём диссертации составляет  \ref*{TotPages}~страницу
%с~\totalfigures{}~рисунками и~\totaltables{}~таблицами. Список литературы
%содержит \total{citenum}~наименований.
%
Полный объём диссертации составляет
\formbytotal{TotPages}{страниц}{у}{ы}{}, включая
\formbytotal{totalcount@figure}{рисун}{ок}{ка}{ков} и
\formbytotal{totalcount@table}{таблиц}{у}{ы}{}.
Список литературы содержит
\formbytotal{citenum}{наименован}{ие}{ия}{ий}.
