\chapter*{Список сокращений и условных обозначений} % Заголовок
\addcontentsline{toc}{chapter}{Список сокращений и условных обозначений}  % Добавляем его в оглавление
% при наличии уравнений в левой колонке значение параметра leftmargin приходится подбирать вручную
\begin{description}[align=right,leftmargin=3.5cm]

\item[\(T\)] множество тем в тематической модели
\item[\(|T|\)] количество тем в тематической модели
\item[\(D\)] множество всех документов
\item[\(|D|\)] количество документов в коллекции
\item[\(W\)] множество всех возможных слов (или токенов)
\item[\(|W|\)] число всех возможных слов (или токенов)

\item[\(n_{wd}\)] количество слов $w$ в документе $d$
\item[\(p_{tdw}\)] вероятность того, что слово $w$ относится к теме $t$ внутри документа $d$
\item[\(n_{wt}\)] оценка количества случаев, когда слово $w$, относится к теме $t$ (по всей коллекции)
\item[\(n_{td}\)] оценка количества слов, относящихся к теме $t$ в документе $d$
\item[\(\phi_{wt}\)] вероятность $w$ в теме $t$
\item[\(\theta_{td}\)] вероятность темы $t$ в документе $d$

\item[ARTM] Additive Regularization of Topic Models
\item[АРТМ] Аддитивная регуляризация тематических моделей
\item[PLSA] Probabilistic Latent Semantic Analysis
\item[LDA] Latent Dirichlet Allocation
\item[EM] Expectation Maximization Algorithm
\end{description}
