
 
  \begin{figure}[ht]
  \small
  \begin{subfigure}[t]{0.39\textwidth}
    $\mbox{A}\ \mbox{group}\ \mbox{of}\ \mbox{\textcolor{my-red}{astronomers}}\ \mbox{managed}$\\
    $\mbox{to}\ \mbox{\textcolor{my-pink}{detect}} \mbox{a}\ \mbox{\textcolor{my-red}{star}},\ \mbox{\textcolor{my-pink}{orbiting}}$\\ $\mbox{\textcolor{my-pink}{around}}\ \mbox{the}\ \mbox{\textcolor{my-red}{black}}\ \mbox{\textcolor{my-red}{hole}}\ \mbox{at}$\\
    $\mbox{a}\ \mbox{very}\ \mbox{close}\ \mbox{\textcolor{my-pink}{distance}}.$
  \end{subfigure}
  ~
  \small
  \begin{subfigure}[t]{0.39\textwidth}
  $\mbox{Over}\ \mbox{the}\ \mbox{course}\ \mbox{of}\ \mbox{a}\ \mbox{single}\ \mbox{busy}\ \mbox{week,}\ \mbox{I}\ \mbox{was}\ \mbox{invited}\ \mbox{to}\ \\ \mbox{an}\ \mbox{\textcolor{my-red}{astronomers'}}\ \mbox{conference},\ \mbox{I}\ \mbox{had}\ \mbox{interviewed}\ \\
  \mbox{a}\ \mbox{famous}\ \mbox{pop}\ \mbox{\textcolor{my-red}{star}},\ \mbox{I}\ \mbox{wrote}\ \mbox{an}\ \mbox{opinion}\ \mbox{piece}\ \\
  \mbox{about}\ \mbox{\textcolor{my-red}{black}}\ \mbox{community}\ \mbox{and}\ \mbox{contributed}\ \\ \mbox{to}\ \mbox{an}\ \mbox{editorial}\ \mbox{on}\ \mbox{a}\ \mbox{\textcolor{my-pink}{massive}}\ \mbox{budget}\ \mbox{\textcolor{my-red}{hole}}.$
  \end{subfigure}
  
  \vspace{0.35cm}
  
  \[
  t = \mbox{Black Holes} = 
      \{\mbox{\textcolor{my-red}{black}}, \mbox{\textcolor{my-red}{hole}}, \mbox{\textcolor{my-red}{star}},
      \mbox{\textcolor{my-red}{astronomer}}, \mbox{\textcolor{my-pink}{orbiting}}, 
      \mbox{\textcolor{my-pink}{distance}}, \mbox{\textcolor{my-pink}{detect}},
      \mbox{\textcolor{my-pink}{massive}}, \ldots\}
  \]
\captionsetup{justification=raggedright,singlelinecheck=false,format=hang}  
\caption{Пример, иллюстрирующий, почему в некоторых случаях меры когерентности, основанные на верхних токенах, могут действовать плохо. Представлены два фрагмента текста. Связанные с темой <<Черные дыры>> слова выделены цветом (красный~---~ сильно связан с темой, розовый~---~ частично связан). Оба фрагмента содержат верхние слова темы примерно на одних и тех же местах в тексте; промежутки между ними заполнены либо слаботематичными словами (левый фрагмент), либо словами, относящимисся к другой теме (правый фрагмент).\\ Когерентности, основанные на верхних токенах, <<считают>> что оба фрагмента~--- когерентные темы, хорошо описывающие данную коллекцию. Предлагаемые внутритекстовые меры качества учитывают поведение тем внутри всех слов, и поэтому имеют возможность определить правый фрагмент как имеющий более низкое качество.  }
  \label{plot:ideal-tm}
\end{figure}